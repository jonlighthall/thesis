 The Helical Orbit Spectrometer (HELIOS) at Argonne National Laboratory is the first implementation of a radical new  concept for measuring nuclear reactions.  
 Direct nuclear transfer reactions are %provide
   powerful tools for studying the properties of the atomic nucleus. A   traditional example %of which
    is the neutron-transfer reaction ($d$,$p$), wherein an accelerated beam of deuterons $d$ bombards a heavy target.  The incoming deuteron transfers a neutron to the target nucleus and the outgoing proton $p$ is detected to study the properties of the residual heavy nucleus.  A new frontier of nuclear reaction studies involving short-lived exotic  nuclei---which are unsuitable for use as targets---is being made available through the development of radioactive ion beam facilities.

In measurements made with radioactive beams, the role of the beam and target are reversed, with a heavy-ion beam
bombarding a light target.  In this regime of ``inverse kinematics,''
the center-of-mass system %of the reaction 
has a substantial velocity in the
laboratory frame,  
the consequence of which %particles being emitted from a frame with a substantial velocity
 is that % the reaction 
the energy of the emitted light
ion is highly angle-dependent.  A successful measurement made in inverse kinematics thus requires a detector system with excellent resolution.
HELIOS is a new approach to detecting the charged light ion reaction products which addresses the technical challenges of studying reactions in inverse kinematics.

The HELIOS spectrometer is based on a large-bore superconducting solenoid %(from an MRI medical scanner)
 which uses a slender po\-si\-tion-\-sen\-si\-tive detector array to measure the energy and position of charged particles along the solenoid axis.  This dissertation describes the %technical
 advantages of the HELIOS concept as they relate to measurements made in inverse kinematics.  The technical specifications of the spectrometer are described in detail.  Three reaction measurements are discussed: the measurement used to commission HELIOS, the $^{28}$Si($d,p$)$^{29}$Si reaction; the first experimental results from HELIOS, the $^{12}$B($d,p$)$^{13}$B measurement; and an important measurement which is planned to be made with HELIOS, the $^{132}$Sn($d,p$)$^{133}$Sn reaction.  Each of these measurements is compared to past measurements made with other detector systems. 